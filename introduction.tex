\subsection*{11/3/24}\label{Sec: 11/3/24}
\begin{Nt}\label{Nt: Vector spaces}
    All vector spaces are real and finite dimensional unless otherwise stated.
\end{Nt}

\begin{Dfn}\label{Dfn: Linear symplectic form}
    A bilinear form $\Omega:V\times V\to \mathbb{R}$ on a vector space $V$ is a \emph{linear symplectic form} if it is
    \begin{enumerate}[label=\DfnLbl]

        \item skew-symmetric, i.e. $\Omega(v,w) = -\Omega(w,v) \quad \forall \ v,w\in V$;

        \item non-degenerate, i.e. $\Omega(v,w) = 0 \quad \forall \ v\in V \implies w=0$.
    \end{enumerate}
\end{Dfn}

\begin{Exp}\leavevmode
    \begin{enumerate}[label=\ExpLbl]

        \item Consider\footnote{Whenever order is needed for a basis, I write them as ordered bases (i.e. between parenthesis) instead of unordered sets, and this is important to me!} $V=\mathbb{R}^2, B = (e_1, e_2)$. Then\footnote{The notation $[\Gamma]_\alpha^\beta$ indicates the matrix representation of the bilinear form $\Gamma$ which takes column vectors $[w]_\alpha$ from the right represented, which are with an ordered basis $\alpha$, and row vectors $([v]_\beta)^T$ from the left, which are represented with an ordered basis $\beta$ and transposed, and computes $\Gamma(v,w)$.} $[\Omega]_B^B=\big(\begin{smallmatrix} 0 &1 \\ -1 &0 \end{smallmatrix}\big)$ represents a linear symplectic form on $\mathbb{R}^2$.

        \item Consider $V=\mathbb{R}^{2n}, B = (e_1,\dots, e_n,f_1,\dots,f_n)$. Then
            \[
                [\Omega]_B^B = \begin{pmatrix} 0 &\vline &I_n \\ \hline -I_n &\vline &0\end{pmatrix}
            \]
            represents a linear symplectic form on $\mathbb{R}^{2n}$. Moreover, we have that
                \begin{align}\label{eq: Symplectic basis}
                    \Omega(e_1,e_j)&=0, \nonumber \\
                    \Omega(f_i,f_j) &= 0, \\
                    \Omega(e_i,f_j) &= \delta_{ij} = -\Omega(f_j,e_i) \nonumber.
                \end{align}
    \end{enumerate}
\end{Exp}

\begin{Dfn}\label{Dfn: Symplectic vector space}
    The pair $(V,\Omega)$ is a \emph{(real) symplectic vector space}. If it has a basis $B = (e_1,\dots, e_n,f_1,\dots,f_n)$ satisfying the relations given by the equations in (\ref{eq: Symplectic basis}), we say that $B$ is a \emph{symplectic basis} of $(V,\Omega)$.
\end{Dfn}

\begin{Rmk}\label{Rmk: Representation of symplectic forms}
    If $B$ is an ordered basis of $V$, then $[\Omega]_B^B$ is antisymmetric and invertible.
\end{Rmk}

\begin{Exp}
    Let $W$ be a vector space with dual $W^\ast$ and $V=W\oplus W^\ast$. Note there is an isomorphism of vector spaces $W\oplus W^\ast\xrightarrow[]{\sim} W^\ast\oplus W, (w,f)\mapsto (f,-w)$. Then,
    \begin{align*}
        \Omega:V\times V &\to R, \\
        ((w_1,f_1),(w_2,f_2)) &\mapsto f_2(w_1) - f_1(w_2)
    \end{align*}
    is a linear symplectic form on $V$.
\end{Exp}

\begin{Lmm}
    If $(V,\Omega)$ is a symplectic vector space, then ${dim}(V) \equiv 0 \mod 2$.

    \begin{proof}

        If $A$ represents $\Omega$, then
        \begin{align*}
            \text{det}(A) &= \text{det}(-A^T) \\
                          &= \text{det}(-A) \\
                          &= (-1)^{\text{dim}(V)}\text{det}(A).
        \end{align*}
    \end{proof}
\end{Lmm}

\begin{Dfn}\label{Dfn: Symplectic, isotropic, coisotropic and Lagrangian subspaces}
    Let $(V,\Omega)$ be a symplectic vector space. A linear subspace $U\subseteq V$ is
    \begin{enumerate}[label=\DfnLbl]

        \item \emph{symplectic} if $\Omega\mid_U$ is a linear symplectic form on $U$ (non-degeneracy is sufficient);

        \item \emph{isotropic} if $\Omega\mid_U=0$;

        \item \emph{coisotropic} if $\Omega(u,v)=0 \quad \forall \ u\in U \implies v\in U$;

        \item \emph{Lagrangian} if it is both isotropic and coisotropic.
    \end{enumerate}
\end{Dfn}

\begin{Exp}\label{Exp: Symplectic, isotropic, coisotropic and Lagrangian subspaces}
    Let $V=\mathbb{R}^{2n}, B=(e_1,\dots,e_n,f_1,\dots,f_n)$.

    \begin{enumerate}[label=\ExpLbl]

        \item $\langle \{e_1, f_1\} \rangle$ is a symplectic subspace which is neither isotropic nor coisotropic (thus neither Lagrangian).

        \item $\langle \{e_1,\dots,e_n\} \rangle$ is a Lagrangian subspace which is not symplectic.

        \item $\{0\}$ and $\langle \{e_1,\dots,\hat{e_k},\dots,e_n\} \rangle$, where $\hat{e_k}$ indicates the exclusion of the $k$-th vector of the canonical ordered basis from the set, are isotropic subspaces which are not coisotropic, where the first one is trivially symplectic.

        \item If $I,J\subseteq \{1,\dots,n\}$ and $\langle \{e_i\}_{i\in I}\cup\{f_j\}_{j\in J} \rangle$ is
            \begin{itemize}

                \item symplectic if, and only if, $I=J$;

                \item isotropic if, and only if, $I\cap J=\varnothing$;

                \item coisotropic if, and only if, $I \cup J = \{1,\dots,n\}$ \footnote{I'm pretty sure of this characterization. But keep this note until we get any kind of confirmation.};

                \item Lagrangian if, and only if, $I=J^c$.
            \end{itemize}
    \end{enumerate}
\end{Exp}

\subsection*{12/3/24}\label{Sec: 12/3/24}

\begin{Rmk}\label{Rmk: Linear isomorphism}
    Let $V$ be a real finite dimensional vector space with dual $V^\ast$ and $\Omega:V\times V\to \mathbb{R}$ a bilinear form. Then, we have the linear isomorphism\footnote{I believe we need to ask for $v\neq 0$, although the professor didn't mention it.}
\begin{align*}
    \Omega^\# = \tilde{\Omega}:V&\to V^\ast, \\
    v&\mapsto \Omega(v,\cdot).
\end{align*}
$\Omega$ is symplectic if and only if
\begin{enumerate}[label=\RmkLbl]

    \item $\Omega^\#$ is antiselfdual, i.e. $(\Omega^\#)^\ast = -\Omega^\#$, and
    \item $\Omega^\#$ is injective (or, equivalently, an isomorphism).
\end{enumerate}
\end{Rmk}

\begin{Dfn}\label{Dfn: Symplectic orthogonal}
    Let $(V,\Omega)$ be a symplectic vector space and $U\subseteq V$ a linear subspace. The \emph{symplectic orthogonal} or \emph{symplectic annihilator} of $U$ is
    \[
    U^\Omega := \{v\in V \mid \Omega(u,v)=0 \quad \forall \ u\in U\}.
    \]
\end{Dfn}

\begin{Prp}[]\label{Prp: }
    Let $(V,\Omega)$ be a symplectic vector space and $U\subseteq V$ a linear subspace. Then,
    \[
    \text{dim}(U) + \text{dim}(U^\Omega) = \text{dim}(V) \quad \text{and} \quad (U^\Omega)^\Omega = U.
    \]
    \begin{proof}

        The first part follows from the fact that $\Omega^\#$ is an isomorphism. Moreover, the inclusion $U\subseteq (U^\Omega)^\Omega$ follows from Definition \ref{Dfn: Symplectic orthogonal} and the equality follows by noticing their dimensions are equal.
    \end{proof}
\end{Prp}

\begin{Rmk}\label{Rmk: Characterization of isotropy and coisotropy}
    Let $U$ be a linear subspace of a symplectic vector space $(V,\Omega)$. Then $U$ is
    \begin{itemize}

        \item symplectic if, and only if $V = U\oplus U^\Omega$;

        \item isotropic if, and only if, $U\subseteq U^\Omega$;

        \item coisotropic if, and only if, $U\supseteq U^\Omega$;

        \item Lagrangian if, and only if, $U= U^\Omega$.
    \end{itemize}
\end{Rmk}

\begin{Exc}
    Prove that
    \begin{enumerate}[label=\ExcLbl]

        \item $U$ is symplectic if, and only if, $U^\Omega$ is symplectic;

        \item $U$ is isotropic if, and only if, $U^\Omega$ is coisotropic;

        \item $(U\cap W)^\Omega = U^\Omega + W^\Omega$.
    \end{enumerate}
\end{Exc}

\begin{Prp}[]\label{Prp: }
    A symplectic vector space $(V,\Omega)$ with $\text{dim}(V)=2n$ has a symplectic basis.

    \begin{proof}

        We will prove this by induction, taking as basis $\text{dim}(V)=2$. Let $e_1\neq 0$. Then there exists $v\in V$ such that $\Omega(e_1,v)\neq 0$. By the Gram-Schmidt orthonormalization process, it follows that $\{e_1, \frac{v}{(e_1,v)}\}$ is a symplectic basis.

        Assume the Proposition holds for $\text{dim}(V)=2n$. Let $v,v'\in V$ be such that $\Omega(v,v')\neq 0$. Then $(S := \langle v,v' \rangle, \Omega\mid_S)$ is a symplectic space of dimension 2. Note that $(S^\Omega, \Omega\mid_{S^\Omega})$ is a symplectic vector space of dimension $2n$. Since $V=S\oplus S^\Omega$, where both subspaces have symplectic bases due to the induction hypothesis, it follows that $(V,\Omega)$ has a symplectic basis.
    \end{proof}
\end{Prp}

\begin{Rmk}\label{Rmk: Dimension of Lagrangian subspaces}
    If $L$ is a Lagrangian subspace of a symplectic vector space $(V,\Omega)$, then $\text{dim}(L) = \frac{\text{dim}(V)}{2}$.
\end{Rmk}

\begin{Prp}[Lagrangian split]\label{Prp: Lagrangian split}
    Let $(V,\Omega)$ be a symplectic vector space. Then, there exist Lagrangian subspaces $L,L'$ such that $V=L \oplus L'$.

    \begin{proof}

        \begin{Exc}
        Write the proof; the idea is to show that you can find a maximal (with respect to dimension) isotropic subspace.
        \end{Exc}

    \end{proof}
\end{Prp}

\begin{Rmk}\label{Rmk: The Lagrangian split is canonical}
    Recall that linear maps $\varphi:V\to V$ induce a map between bilinear forms via
    \[
    \varphi^\ast(\Omega(v,v')) = \Omega(\varphi(v),\varphi(v')).
    \]
    The Lagrangian split $V=L\oplus L'$ of Proposition \ref{Prp: Lagrangian split} is canonical in the sense that there exists a canonical isomorphism $\varphi:V\to L\oplus L'$ such that $(\varphi^{-1})^\ast\Omega$ is the canonical symplectic form on $L\oplus L'$, i.e.
    \[
    \Omega(v_1+v_1', v_2+v_2') = \Omega(v_1,v_2') - \Omega(v_2,v_1').
    \]

\end{Rmk}

\begin{Dfn}\label{Dfn: Topological manifold}
    $M$ is a \emph{topological manifold} if it is a topological space such that
    \begin{itemize}

        \item $\forall \ p\in M$, there exists a neighborhood $V$ of $p$ that is homeomorphic to an open set in $\mathbb{R}^n$;

        \item it is Hausdorff, i.e. for any two points we can find a neighbourhood for each such that they are disjoint;

        \item it satisfies the second countability axiom, i.e. there exists a countable basis.
    \end{itemize}
\end{Dfn}

\begin{Exp}\leavevmode
    \begin{enumerate}[label=\ExpLbl]

        \item The torus $T^2=S^1\times S^1$, obtained via the identification \simple{Add identification!}.

        \item The Klein bottle $K^2$, obtained via the identification $(x,y)\sim (x+1,y)\sim (1-x,y+1)$ in $[0,1]\times[0,1]\subseteq \mathbb{R}^2$.

        \item The projective plane, obtained via the identification $(x,y)\sim (x+1,1-y)\sim (1-x,y+1)$ in $[0,1]\times[0,1]\subseteq \mathbb{R}^2$.
    \end{enumerate}
\end{Exp}

\begin{Rmk}\label{Rmk: Differentiable manifold}
    Recall that $M$ is a differential manifold if it is a topological manifold of dimension $n$ with an atlas, which is a collection of charts $\{U_\alpha,\varphi_\alpha\}_{\alpha\in A}$ such that
    \begin{enumerate}[label=\RmkLbl]

        \item $U_{\alpha\in A} \varphi_\alpha(U_\alpha) = M$,

        \item $W = \varphi_\beta(U_\beta)\cap\varphi_\alpha(U_\alpha)\neq 0$, where $\varphi_\beta^{-1}\varphi_\alpha$ and $\varphi_\alpha\varphi_\beta^{-1}$ are of class $C^\infty$,

        \item $U_{\alpha\in A} \varphi_\alpha(U_\alpha)$ is maximal.
    \end{enumerate}
\end{Rmk}

\begin{Exp}\label{Exp: Differentiable manifolds}
    \simple{Add examples and complete notes for the last half hour of the class.}
\end{Exp}

\subsection*{18/3/24}\label{Sec: 18/3/24}
