\documentclass[misc]{subfiles}
\begin{document} 

\section*{Symplectic Manifold}

\begin{Dfn}
    Given a differential manifold $M$ we say that a $2$--form $\omega$ is symplectic if it is
    \begin{enumerate}[label=\DfnLbl]

        \item closed, i.e. $d\omega=0$;

        \item non-degenerate, i.e. $\omega(u,v)=0 \quad \forall \ v\in M \implies u=0$.

        \item for each $p$ we have that $\omega_p$ is a symplectic form on $T_pM$.
    \end{enumerate}

    We call the couple $(M,\omega)$ symplectic manifold.
\end{Dfn}

\begin{Prp}
    A symplectic manifold always has even dimension.
\end{Prp}

\begin{Exp}
    $(\mathbb{R}^{2n},\omega_{std})$ with coordinates $x_1,\dots,x_n,y_1,\dots,y_n$. $\omega=\sum_i dx_i \wedge dy_i$.

    For each $p$, $\omega_p$ is given by: $\left( (\frac{\partial}{\partial x_1})_p, \dots, (\frac{\partial}{\partial x_n})_p, (\frac{\partial}{\partial y_1})_p, \dots, (\frac{\partial}{\partial y_n})_p\right)$. It is symplectic on $T_pM$.

    We can do the same considering $\mathbb{C}^n$ with $\omega=\frac{1}{2} \sum_i dz_i \wedge d\bar{z_i}$.
\end{Exp}

\begin{Exp}\
    Take now the sphere $S^2$ as a subset of $\mathbb{R}^3$. We know that $T_pS^2$ is given by $\{p\}^\perp$.

    Now we define the symplectic form $\omega$ in the following way $\omega_p(u,v)=<p; v \times u>$. In fact it is closed is because it is a $2$--form in a dimension $2$ manifold and it is non degenerate becuase $v \times u$ always has the direction of $p$ as they are in $T_pS^2$.\

    This $\omega$ is a volume form for $S^2$.
\end{Exp}

\begin{Exc}
    The above definition is an implicit one; we can define the same form in coordinates taking the coordinates $\theta, z$. In this case the form can be expressed as $\omega= d\theta \wedge dz$.

    Also how will the form change if we take the sphere of radius $R$?
\end{Exc}

\begin{Dfn}
    Given two symplectic manifolds $(M_1,\omega_1)$ and $(M_2,\omega_2)$ we say that a \emph{symplectomorphism} is a diffeomorphism $g$ such that $g^\star\omega_2=\omega_1$.

    As before, the dimension is the only invariant, but now only in a local scope.
\end{Dfn}

\simple{Add all the cotangent bundles!}

\begin{Prp}
    $S^2$ is the only sphere that can be symplectic, all the others have a trivial second order cohomology group, so they cannot have a symplectic form.
\end{Prp}

\begin{Prp}
    All the symplectomorphisms preserve the areas but the converse is not true nor a banal proof. It was worthy of a Field's medal.
\end{Prp}

\subsection*{19/03/2024}

Last class we considered a manifold $X$ of dimension $n$ and saw that
\[
M=T^\ast:=\{(x,\xi^\ast), x\in X, \xi^\ast\in T_x^\ast X\}
\] 
is a symplectic manifold of dimension 2n. If $(x_1,\dots,x_n)$ is a coordinate system on $X$, then $\{(dx_1)_x,\dots,(d_n)_x)\}$ is a basis for $T_x^\ast X$. Also, $\xi^\ast=\sum_{i=1}^n y_i(dx_i)_x$ and $(x_1,\dots,x_n,y_1,\dots,y_n)$ are coordinates on $T^\ast X$, with
\begin{align*}
    \omega_\text{can} &= \sum_{i=1}^n dx_1\wedge dy_1, \\
                      &= \lambda_\text{can} = \sum_{i=1}^n y_i dx_i, \\
                      &= -d\lambda_\text{can}.
\end{align*}

\simple{Part of the reminder is missing (she erased it too fast).}

Let $X_1,X_2$ be differentiable manifolds of dimension $n$,
\begin{align*}
    M_1 &= (T^\ast X_1, -d\lambda_1), \\
    M_2 &= (T^\ast X_2, -d\lambda_2),
\end{align*}
where $\lambda_i$ is the canonical Liouville term on $X_i$ for $i\in\{1,2\}$. Let $f:X_1\to X_2$ be a differomorphism. Then
\begin{align*}
    f_\mathbb{X}: M_1 &\to M_2, \\
    p_1 &\mapsto p_2,
\end{align*}
where $p_1 = (x_1,\xi_1^\ast)$ and $p_2 = (f(x_1), \xi_2^\ast)$, where $\xi_2^\ast\in T_{f(x_1)}^\ast X_2$. We have that
\begin{align*}
    df:TX_1 &\to TX_2 \\
    df_{x_1}: T_{x_1}X_1 &\to T_{f(x_1)}X_2, \\
    (df_{x_1})^\ast: T^\ast_{f(x_1)}X_2 &\to T^\ast_{x_1}X_1.
\end{align*}
Since $f$ is in particular a bijection then, defining $x_2:=f(x_1)$, we have that
\begin{align*}
    df^{-1}_{x_1}:T_{x_2}X_2 &\to T_{x_1}X_1, \\
    (df^{-1}_{x_1})^\ast:T_{x_1}^\ast X_1 &\to T_{x_2}^\ast X_2, \\
    \xi_1^\ast &\mapsto (df^{-1}_{x_2})^\ast(\xi_1^\ast) =: \xi_2^\ast.
\end{align*}
Moreover, we have the following commutative diagram
\begin{equation}
    \begin{tikzcd}
        M_1 \arrow[]{d}[swap]{\pi_1} \arrow[]{r}[]{f_\mathbb{X}} &M_2 \arrow[]{d}[]{\pi_2} \\
        X_1 \arrow[]{r}[swap]{f} &X_2.
    \end{tikzcd}
\end{equation}

\begin{Prp}[]\label{Prp: Modular forms implies associated 2-forms}
    Let $f_\mathbb{X}:M_1\to M_2$ be a diffeomorphism be such that $f_\mathbb{X}^\ast(\lambda_2)=\lambda_1$. Then $f_\mathbb{X}^\ast(-d\lambda_2) = -d\lambda_1$.

    \begin{proof}
    
        Let $p_1 = (x_1, \xi_1^\ast)\in T^\ast X_1$ and $p_2 = f_\mathbb{X}(p_1) = (f(x_1), \xi_2^\ast)$.
        \begin{align*}
            (f_\mathbb{X})^\ast((\lambda_2)_{p_2}) &= (f_\mathbb{X}^\ast)_{p_1}((d\pi_2)_{p_2}^\ast\xi_2^\ast) \\
                                                   &= d(f\pi_1)_{p_1}^\ast \xi_2^\ast \\
                                                   &= (d\pi_1)^\ast_{p_1} f^\ast_\mathbb{X}(\xi_2^\ast) \\
                                                   &= (d\pi_1)_{p_1}^\ast \xi_1^\ast \\
                                                   &= (\lambda_1)_{p_1}.
        \end{align*}
    \end{proof}
\end{Prp}

\begin{Cor}[]\label{Cor: Diffeomorphism induces symplectomorphism}
    If $X_1$ and $X_2$ are diffeomorphic, then $T^\ast X_1$ and $T^\ast X_2$ are symplectomorphic.
\end{Cor}

\begin{Exp}
    Let $(M_1,\omega_1)$ and $(M_2,\omega_2)$ be symplectic manifolds. Then we have a product $M_1\times M_2$ with projections
    \begin{center}
        \begin{tikzcd}
            &M_1\times M_2 \arrow[]{dl}[swap]{p_1} \arrow[]{dr}[]{p_2} \\
            M_1 &&M_2.
        \end{tikzcd}
    \end{center}
    Then, $\omega_{a,b}:= ap_1^\ast\omega_1 + bp_2^\ast\omega_2$ is a symplectic form on $M_1\times M_2$ for all $a,b\in \mathbb{R}$. In fact,
    \begin{align*}
        d\omega_{a,b} &= d(ap_1^\ast\omega_1 + bp_2^\ast\omega_2) \\
                      &= ad(p_1^\ast\omega_1) + bd(p_2^\ast\omega_2) \\
                      &= ap_1^\ast d\omega_1 + bp_2^\ast d\omega_2 \\
                      &= 0,
    \end{align*}
    i.e. the closeness of $\omega_{a,b}$ is induced from that of $\omega_1$ and $\omega_2$. Similarly, the other properties of induced.
\end{Exp}

\begin{Exp}
    $\mathbb{R}^{2n}, \mathbb{C}^n, S^2$. Consider $S^2\times \mathbb{R}^2$ with the form $\omega = 2p_1^\ast\omega_1 - p_2^\ast\omega_2$. On $S^2$, $\omega_p(u,v) = \langle p, u\times v \rangle$, where $p=(x_1,x_2,x_3)\in S^2\subseteq \mathbb{R}^3$ is such that $\sum_{i=1}^3 x_1^2 = 1$, and $u=(u_1,u_2,u_3),v=(v_1,v_2,v_3)\in T_pS^2\simeq T_p^\ast S^2$. Explicitely,
    \[
    \omega_p(u,v) = x_1(u_2v_3 - v_2u_3) + x_2(u_3v_1 - u_1v_3) + x_3(u_1v_2 - v_1u_2).
    \] 
    On the other hand, we know that $\omega_p = \sum_{i,j} a_{ij} dx_i\wedge dx_j$, where
    \[
    a_{ij} = \bigg( \frac{\partial}{\partial x_i}, \frac{\partial}{\partial x_j}\bigg).
    \] 
    In particular, $a_{ii}=0$ for all $i$. We can thus calculate
    \[
    \omega_p = x_3dx_1\wedge dx_2 - x_2dx_1\wedge dx_3 + x_1dx_2\wedge dx_3 
    \] 
\end{Exp}

\simple{Complete the notes for the last half hour of the class.}

\subsection*{25/03/2024}

\begin{Rmk}
    Given two symplectic manifolds $(M_1,\omega_1)$ and $(M_2,\omega_2)$ we can endorse the product $M_1 \times M_2$ with a symplectic form $\omega_{ab}$ of the form:
    \[
    \omega_{ab}=ap^\ast_1\omega_1+bp^\ast_2\omega_2,
    \]
    where $p_i$ is the projection over $M_i$ and $a,b \in \mathbb{R} \setminus \{0\}$.

    In fact fow each $u=(u_1,u_2), v=(v_1,v_2)$ we have that:
    \begin{align*}
        \omega_{a,b}(u,v)   &= ap_1^\ast\omega_1(u,v) + bp_2^\ast\omega_2(u,v) \\
                            &= a\omega_1(u_1,v_1) + b\omega_2(u_2,v_2),
    \end{align*}
    but if we suppose that for each $v$ we have $=0$ than choosing $v=(v_1,0)$ we get $u_1=0$ for the non--degeneracy of $\omega_1$ and vice versa for $u_2$ but if and only if $a \neq 0$ and $b \neq 0$.
\end{Rmk}

\begin{Dfn}
    Given a symplectic manifold $(M,\omega)$ and a submanifold $W$, we say that $W$ is symplectic if and only if $\omega_{|W}$ is a symplectic form, and then $\dim W$ is even.
\end{Dfn}

\begin{Rmk}
    This condition pass to the tangent space, $T_pW$ si a symplectic subvector space of $T_pM$ for each point $p$.
\end{Rmk}

\begin{Dfn}
    Given a symplectic manifold $(M,\omega)$ and a submanifold $L$, we say that $L$ is lagrangian if and only if one of the following three equivalent characterization is true:

    \begin{enumerate}[label=\DfnLbl]

        \item for each point $p$ $T_pL \subset T_pM$ is lagrangian as vector space;

        \item $\omega_{|L} \equiv 0$ and $\dim L = \frac{\dim M}{2}$;

        \item $i^\ast\omega \equiv 0$ and $\dim L= \frac{\dim M}{2}$.
    \end{enumerate}
\end{Dfn}

\begin{Exc}
    Take $\mathbb{R}^{2n}$ with form $\omega=\sum_{i \neq j} dx_i \wedge dx_j$ then this is not a symplectic manifold because $\omega$ degenerates.
\end{Exc}

\begin{Exp}
    Take the cotangent bundle $T^\star M$ with the canonical form, what are its lagrangian submanifold?

    We can take $\mathbb{R}^{2n}$ as $T^\star\mathbb{R}^n$ and the standard form is eqivalent to the Liouville ones, the $-$ is only because the order of the element in the basis is used different inside the two forms.

    In general we have, as lagrangian submanifolds:
    \begin{enumerate}[label=\ExpLbl]
        \item the zero section $\{(m,0)\}$;

        \item the fibers $T_xM$;

        \item some other sections, in particular all the ones such that the relative $1$--form $\mu$ is a closed one, so $d\mu=0$.
    \end{enumerate}
\end{Exp}

\begin{Exp}
    Take now the sphere $S^2$ with the standard form. Its lagrangian submanifolds are given by the equator, the meridians and the parallels. For this last one we have to see them like the earth, so line that connects the north pole to the south pole, they will fix $\theta$.

    The symplectic structure breaks something in the symmetry of the sphere, only the meridians are lagrangian, while all the other geodetics are not.
\end{Exp}

\begin{Prp}
    Take a diffeomorphism $\phi: m_1 \rightarrow M_2$, it is a symplectomorphism if and only if the graph $\Gamma_\phi$ is a lagrangian submanifold inside $(M_1 \times M_2, \omega_{1,-1})$.

    \begin{proof}
        Let be $\gamma: M_1 \rightarrow M_1 \times M_2$ the map that sends $x$ into $(x,\phi(x))$. now we have:
        \begin{align*}
        \Gamma_\phi \text{ is lagrangian}   &\iff \gamma^\ast\omega_{1,-1}=0 \\
                                            &\iff \gamma^\ast p_1^\ast \omega_1-\gamma^\ast p_2^\ast \omega_2=0 \\
                                            &\iff (p_1 \circ \gamma)^\ast \omega_1-(p_2 \circ \gamma)^\ast \omega_2=0 \\
                                            &\iff \forall x \in M_1 \text{   } \omega_1(x)-\omega_2(\phi(x))=0 \\
                                            &\iff \omega_1 = \phi^\ast \omega_2 \\
                                            &\iff \phi \text{ is a symplectomorphism}.
        \end{align*}
    \end{proof}
\end{Prp}

\begin{Exp}
    Take $\{x^0\} \times M_2 = W_2$ inside $M_1 \times M_2$, it is lagrangian? First of all we ahve to suppose $\dim M_1 = \dim M_2$ but it is not sufficient. For example $\mathbb{R}^2 \times \mathbb{R}^2$ with form $\omega=dx_1 \wedge dx_2 + dy_1 \wedge dy_2$, if we take $\{(x^0_1,x^0_2)\} \times \mathbb{R}^2$ we get a symplectic submanifold.
\end{Exp}

\begin{Exp}
    Take $M$ as the product of $n$ copies of $S^2$ end use $E_1$ to denote the equator, as form we take $\omega=\sum_i a_1 p^\ast_i \omega_i$ where $\omega_i$ is the standard form on the $i$--th sphere.

    Each sphere is a symplectic submanifold, once we fix any point on the other, the same is true for each families of sphere, fixing a pint in each sphere that we want to eliminate.

    If we take the product of all the equators we get a lagrangian submanifold. Taking only some equators, like two, and fixing a point in the other sphere will give us a submanifold that is nor symplectic nor lagrangian for a dimension problem.
\end{Exp}

\begin{Dfn}
    Given a manifold $M$ and a function $\psi: M \times \mathbb{R} \rightarrow M$ we can define the family $\{\psi_t\}$ as $\psi_t(p)=\psi(p,t)$. We will call it an isotopy if they are all diffeomorphisms and $\psi_0$ is the identity over $M$.

    From an isotopy we get a family of vector field fiven by $\frac{d\psi_t}{dt}=X_t \circ \psi_t$. Also if $M$ is compact, then we can define a family of vector field $X_t$ with compact support and find an isotopy that respects the above condition.

    If $X_t=X$ we get the flow of $X$.
\end{Dfn}

\simple{She also defined the integral curve and the lie derivative, I hope that she reprise them in the next lessons because she went really fast}

\section*{25/3/24}\label{Sec: }

    If $S\subseteq \mathbb{R}^3$ is an orientable surface then $\omega\in\Omega^2(M)$ is the volume form. $\omega$ is symplectic in $M$.

\begin{Exp}
    \simple{Add example of the symplectic form given by a torus.}
\end{Exp}

\begin{Thm}
    Let $M$ be a symplectic manifold and $\{\psi_t\}$ an isotopy. Let $\omega_t, t\in\mathbb{R}$ be a family of $d$-forms that depends mostly on $t$ Then
    \[
        \frac{d}{dt}\psi_t^\ast\omega_t = \psi_t^\ast\bigg( \mathcal{L}_{x_t}\omega_t + \frac{d\omega_t}{dt}\bigg).
    \] 

    \begin{proof}
    
        Not given.
    \end{proof}
\end{Thm}

\begin{Thm}[Moser]
    Let $\omega_t$ be a family of symplectic forms such that $\frac{d}{dt}\omega_t = d\omega_t$ (note that $\frac{d}{dt}[\omega_t]=0$). Then there exists a family of diffeomorphisms $\psi_t\in\text{Diff}(M)$ such that $\psi_t^\ast\omega_t = \omega_0$.

    \begin{proof}
    
        Not given.
    \end{proof}
\end{Thm}

\begin{Exp}
    Consider the compact manifold $S^2$ with the symplectic forms $\omega_0=d\theta\wedge dz$ and $\omega_1=rd\theta\wedge dz$. Then $[\omega_0]=[\omega_1]$. If we define
    \begin{align*}
        \omega_t &= (1-t)d\theta\wedge dz + trd\theta\wedge dz \\
                 &= (1-t+tr)d\theta\wedge dz
    \end{align*}
    for every $t\in[0,1]$. We want a a family of functions $\psi_t:S^2\to S^2$ such that $\psi_t^\ast\omega_t=\omega_0$, i.e.
    \[
    \psi_t^\ast\omega_t(u,v) = \omega_0(d\psi_1(u), d\psi_t(v)),
    \] 
    for all $t\in[0,1]$. %The Prof tried to give a motivating example of the following Lemma because she hadn't prepared the proof but gave up and ended up erasing it.
    
\end{Exp}

\begin{Lmm}
    Let $M^{2n}$ be a compact manifold. Lat $\omega_0,\omega_1$ be symplectic forms on $M$ such that $[\omega_0]=[\omega_1]$. Assume that the 2-form $\omega_t = (1-t)\omega_0+t\omega_1$ is symplectic for all $t\in[0,1]$. Then, there exists an isotopy $\{\psi_t\}$ such that $\psi_t^\ast\omega_t=\omega_0$ for every $t\in[0,1]$.

    \begin{proof}
    
        Not given.
    \end{proof}
\end{Lmm}

\begin{Thm}[Darboux]\label{Thm: Darboux's Theorem}
    Every symplectic form on $M^{2n}$ can be deformed to the extended symplectic form $\omega_0$ on $\mathbb{R}^{2n}$. Equivalently, if $(M^{2n},\omega)$ is a symplectic manifold, then every point $p\in M$ one can define a local chart centered at $p$ with coordinates $(x_1,\dots,x_n,y_1,\dots,y_n)$ such that
    \[
    \omega = \sum_{i=1}^n dx_i\wedge dy_i.
    \] 

    \begin{proof}
    
        Not given (for now?).
    \end{proof}
\end{Thm}

\begin{Rmk}
    It follows from Theorem \ref{Thm: Darboux's Theorem} that the only \textbf{local} symplectic invariant is the dimension.
\end{Rmk}

\begin{Prp}
    Let $Q$ be a submanifold of a symplectic manifold $M$ and let $\omega_0,\omega_1$ be two symplectic forms on $M$ such that $\omega_0\mid_Q=\omega_1\mid_Q$. Then there exist two neighbourhoods $N_0, N_1$ of $Q$ such that there exists a differomorphism $\psi:N_0\to N_1$ such that $\psi_0=\text{Id}$ and $\psi^\ast\omega_1=\omega_0$.
\end{Prp}

\subsection*{Complex structures}\label{Ssec: Complex structures}

\begin{Dfn}\leavevmode
    \begin{enumerate}[label=\DfnLbl]

        \item A \emph{complex structure} on a vector space $V$ is a linear operator $J:V\to V$ such that $J^2=-\text{Id}$.
    
        \item A complex structure $J$ on symplectic a vector space $(V,\Omega)$ is \emph{compatible} with $\Omega$ if 
            \[
            G(u,v) := \Omega(u, Jv) \quad \forall \ u,v\in V
            \] 
            is an inner product in $V$.
    \end{enumerate}
\end{Dfn}

\begin{Exp}
    For $(\mathbb{R}^2,\omega= dz\wedge \overline{d}z), J:\mathbb{C}\to \mathbb{C}, u\mapsto iu$ is a compatible complex structure.
\end{Exp}

\begin{Nt}
    A complex structure $J$ is compatible with a symplectic a symplectic structure $\Omega$ on a symplectic vector space $V$ if, and only if, $J$ is a symplectomorphism. This follows from the fact that, for all $u,v\in V$,
    \begin{align*}
        J^\ast\Omega(u,v) &= \Omega(Ju,Jv) \\
                          &= G(Ju,v) \\
                          &= G(v,Ju) \\
                          &= \Omega(v, JJu) \\
                          &= \Omega(v,-u) \\
                          &= \Omega(u,v).
    \end{align*}
\end{Nt}

\begin{Prp}
    Let $(V,\Omega)$ be a symplectic vector space. Then there exists a complex structure $J$ compatible with $\Omega$. Moreover, given an inner product $\langle \cdot, -\rangle$ on $V$ we can construct a complex structure $J$ on $V$ that is compatible with $\Omega$.

    \begin{proof}
    
        Let $\{e_1,\dots,e_n,f_1,\dots,f_n\}$ be a symplectic basis. Define $J_{e_i}=f_i, J_{f_i}=-e_i$ and the maps
        \begin{align*}
            \phi_1:V&\to V^\ast \\
            v&\mapsto \Omega(v,-), \\ \\
            \phi_2:V&\to V^\ast \\
            v&\mapsto \langle v,-\rangle, \\ \\
            A:V&\to V, \\
            v&\mapsto (\phi_2^{-1}\phi_1)v.
        \end{align*}
        Note that, for all $u,v\in V$,
        \begin{align*}
            \langle Au, v \rangle &= \langle (\phi_2^{-1}\phi_1)(u), v \rangle \\
                                  &= \big(\phi_2\big((\phi_2^{-1}\phi_1)(u)\big)\big) (v) \\
                                  &= (\phi_1(u))(v) \\
                                  &= \Omega(u,v); \\ \\
            \langle A^\ast u, v \rangle &= \langle u, Av \rangle \\
                                        &= \langle Av, u \rangle \tag{Since our vector space is real.} \\
                                        &= \Omega(v,u) \\
                                        &= -\Omega(u,v) \\
                                        &= -\langle Au, v \rangle \\
                                        &= \langle -Au, v \rangle.
        \end{align*}
        Thus, $A^\ast=-A$ and $(A^\ast)^\ast=A$. \simple{Finish this proof and complete the notes for the last half hour of the class.}
    \end{proof}
\end{Prp}

\subsection*{8/4/24}\label{Ssec: 8/4/24}

\simple{Fill in notes for this Monday.}

\subsection*{9/4/24}\label{Ssec: 9/4/24}

\begin{Dfn}\label{Dfn: Poisson bracket}
    Let $(M,\omega)$ be a symplectic manifold. The \emph{Poisson bracket} of $f,g\in\mathbb{C}^\infty(M,\mathbb{R})$ is
    \[
    \{f,g\} := \omega(X_f,X_g).
    \] 
    In particular, $X_{\{f,g\}} = -[X_f,X_g]$.
\end{Dfn}

\begin{Thm}[]\label{Thm: Poisson Jacobi}
    The \hyperref[Dfn: Poisson bracket]{Poisson bracket} satisfies the Jacobi identity.

    \begin{proof}
    \end{proof}
\end{Thm}

\begin{Dfn}\label{Dfn: Poisson algebra}
    A \emph{Poisson algebra} $(P,\{\cdot,-\})$ is a commutative associative algebra equipped with a bracket $\{\cdot,-\}$ that satisfies Leibniz's rule:
    \[
    \{f,gh\} = \{f,g\}h + g\{f,h\}.
    \] 
\end{Dfn}

\begin{Exp}
    If $(M,\omega)$ is a symplectic manifold, then $(C^\infty(M,\mathbb{R}),\{\cdot,-\})$ is a Poisson algebra. With respect to the map
    \begin{align*}
        C^\infty(M,\mathbb{R}) &\to \chi(M), \\
        H &\mapsto X_H,
    \end{align*}
    the Poisson bracket $\{\cdot,-\}$ ``transforms into'' $-[\cdot,-]$, as noted in Definition \ref{Dfn: Poisson bracket}.
\end{Exp}

\begin{Dfn}\label{Dfn: Hamiltonian system}
    A \emph{Hamiltonian system} is a triple $(M,\omega,H)$ where
    \begin{enumerate}
    
        \item $(M,\omega)$ is a symplectic manifold,

        \item $H\in C^\infty(M,\mathbb{R})$, called the \emph{Hamiltonian function}.
    \end{enumerate}
\end{Dfn}

\begin{Exp}
    $(S^2,\omega,H(0,z)=z)$ is a \hyperref[Dfn: Hamiltonian system]{Hamiltonian system}.
\end{Exp}

\begin{Prp}
    Let $(M,\omega,H)$ be a Hamiltonian system. Then $\{f,H\}=0$ if, and only if, $f$ is constant along integral curves of $X_H$.

    \begin{proof}
    
        Note that
        \begin{align*}
            \frac{d}{dt}(f\phi_H^t) &= \frac{d}{dt}((\phi_H^t)^\ast f) \\
                                    &= (\phi_H^t)^\ast\mathcal{L}_{X_H}f \\
                                    &= (\phi_H^t)^\ast df(X_H) \\
                                    &= (\phi_H^t)^\ast (t_{X_f}\omega(X_H)) \\
                                    &= (\phi_H^t)^\ast \omega(X_f, X_H) \\
                                    &= (\phi_H^t)^\ast \{f,H\}.
        \end{align*}
    \end{proof}
\end{Prp}

\begin{Dfn}\label{Dfn: First integrals}
    Let $(M,\omega,H)$ be a \hyperref[Dfn: Hamiltonian system]{Hamiltonian system}.

    \begin{enumerate}[label=\DfnLbl]
    
        \item A \emph{first integral} of the Hamiltonian system is a function $f$ such that $\{f,H\}=0$.

        \item Functions $f_1,\dots,f_n\in C^\infty(M)$ are \emph{independent} if $(df_1)_p,\dots,(df_n)_p$ are linearly independent for all $p$ in a open dense subset of $M$.

        \item $(M,\omega,H)$ is (\emph{completely}) \emph{integrable} if it has $\frac{\text{dim}(M)}{2}$ independent first integrals $f_1=H,f_2,\dots,f_n$ such that $\{f_i,f_j\}=0$ for all $i,j\in\{1,\dots,n\}$.
    \end{enumerate}
\end{Dfn}

\begin{Nt}
    Let $(M,\omega,H)$ be an integrable Hamiltonian system with first integrals $f_1=H,f_2,\dots,f_n$, $f=(f_1,\dots,f_n)$ and $f$ be a regular value for $f$. Then $f^{-1}(c)$ is a Lagrangian submanifold of $M$.
\end{Nt}

\simple{Complete notes from second half of the class.}

\subsection*{16/04/2024}

\begin{Exp}
    Let $M=\mathbb{C}\mathbb{P}^m=\mathbb{P}(\mathbb{C}^{n+1})$, if we take $n=1$ we get $\mathbb{C}^2 \cong \mathbb{R}^4$ and we can take the sphere $S^3$ inside. It is not symplectic but it is contained in $\mathbb{R}^4$ that is symplectic. On the sphere we consider the standard equivalence relation for getting the projective space as a quotient, in this manner the equivalence class of a point is given by two antipodal points on the sphere.

    Now if we see the inclusion of the sphere and its projection, both with the standard symplectic form, we get the Fubini--Study 2--form that it is symplectic.
\end{Exp}

\begin{Thm}[Morden--Werenstain reduction theorem]

    $\omega_{FS}$ is symplectic and $i^\star\omega_{std}=p^\star\omega_{ps}$, where $i$ is the inclusion of the sphere, $p$ is the projection map given by the quotient and the other form are the standard ones.

\begin{proof}[Sketch]
    We take a point $z \in S^{2n-1}$, then the tangent space $T_zS^{2n-1}=T_{p(z)}\mathbb{P}^{n-1} \oplus \langle z \rangle$. So $di^\star\omega_{std}=dp^\star\omega_{ps}$, but $d$ commutes and we get $d\omega_{FS}=0$.
\end{proof}

\end{Thm}

\subsection*{Integrable System}

\begin{Dfn}
    Let $(M,\omega,H)$ and Hamiltonian system, we will call \textit{first integral} a function $f \in C^\infty(M,\mathbb{R})$ such that the Poisson bracket with $H$ are 0: $\{f,H\}=0=\omega(X_f,X_H)$.
\end{Dfn}

\begin{Dfn}
    An Hamiltonian system is called \textit{completely integrable} if exist $n$ function, where the dimension of $M$ is $2n$, such that one of this is equal to $H$ and they are all indipendent, so $\{f_i,f_j\}=0$.

    We can define a local action of $\mathbb{R}^n$ over $M$ as $t \cdot p=\phi_n^{t_n} \circ \circ \phi_1^{t_1}(p)$ where $\phi_i$ is the flow of $X_{f_i}$.
\end{Dfn}

\begin{Prp}
    Thanks to the fact that the elements commutes we have that the action is locally free.
\end{Prp}

\begin{Exp}
    \simple{I'm not really sure that it is an example}
    We want to study the $C$--connected components of a regular level set. As first observation we see that $\mathbb{R}^2$ acts on $C$ with discrete stabilizer $\{0\} \times 2^k$.

    Let $(M,\omega,H)$ a completely integrable Hamiltonian system and let $f=(f_1,\dots,f_n)$. Take now a point $q$ that is a regular value for $f$ and $U$ a open neighbourhood of $q$ composed only by regular value. It exists $V$ such that $f(\overline{V}) \subset U$ compact. If we denote with $g$ the restricition of $f$ to $V$ we get that it has value in $U \times F_q$ where $F_q$ is a compact connected component of $ff^{-1}(q)$ if $U$ and $V$ are sufficientely small.
\end{Exp}

\subsection*{17/04/2024}

\subsection*{30/04/2024}

\begin{Dfn}\label{Dfn: Lie group}
    A \emph{Lie group} is a group $G$ and a manifold such that the multiplication and the inverse are smooth maps.
\end{Dfn}

\begin{Exp}\label{Exp: Lie group}

    The following are Lie groups\footnote{For sets of matrices, the group operation is the corresponding matrix multiplication.}:
    \begin{enumerate}
    
        \item $(\mathbb{R}^n,+)$,

        \item $(\mathbb{R}\setminus\{0\},\cdot)$,

        \item $(\mathbb{R}_+,\cdot)$,

        \item $(S^1 = \{z\in\mathbb{C} \mid |z|=1\},\cdot)$,

        \item $\text{GL}(n,\mathbb{R})$, where the smooth structure is induced when considering $\text{GL}(n,\mathbb{R})\subseteq \mathbb{R}^{n^2}$;

        \item $\text{SU}(2) = \{A\in\text{GL}(2,\mathbb{C}) \mid A (\overline{A})^{t}=\text{Id}\} = \big\{ \big( \begin{smallmatrix} \alpha &\beta \\ -\overline{\beta} &\overline{\alpha} \end{smallmatrix} \big) \mid |\alpha|^2 + |\beta|^2 = 1, \alpha\beta\in\mathbb{C}\big\}$;

        \item $\text{SL}(n,\mathbb{R})$;

        \item $\text{O}(n,\mathbb{R}) = \{A\in\text{Mat}_{n\times n}(\mathbb{R}) \mid A A^t=\text{Id}\}$;

        \item $\text{SO}(n,\mathbb{R}) = \{A\in\text{Mat}_{n\times n}(\mathbb{R}) \mid A A^t=\text{Id}, \text{det}(A)=1\}$;

        \item $\text{U}(n,\mathbb{C}) = \{A\in\text{Mat}_{n\times n}(\mathbb{R}) \mid A A^t=\text{Id}\}$;

        \item $\text{SU}(n,\mathbb{C}) = \{A\in\text{Mat}_{n\times n}(\mathbb{R}) \mid A A^t=\text{Id}, \text{det}(A)=1\}$;

        \item $\text{Sp}(2n,\mathbb{R}) = \{A\in\text{Mat}_{2n\times 2n}(\mathbb{R}) \mid M^t\Omega M=\Omega\}$, where $\Omega = \big(\begin{smallmatrix} 0 &\text{Id} \\ -\text{Id} &0 \end{smallmatrix}\big)$.
    \end{enumerate}
\end{Exp}

\begin{Dfn}\label{Dfn: Lie subgroup}
    Let $G$ be a Lie group. A subgroup $H\le G$ is a \emph{Lie subgroup} if it is also a submanifold of $G$.
\end{Dfn}

\begin{Nt}
    Any closed subgroup of a Lie group is a Lie subgroup.
\end{Nt}

\end{document}
